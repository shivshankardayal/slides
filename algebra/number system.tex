\documentclass[aspectratio=1610]{beamer}
\usepackage{tikz, xcolor, amsmath}
\definecolor{webgreen}{rgb}{.4,.7,.4}
\color{webgreen}
\setbeamercolor{structure}{fg=webgreen} 
\usetheme{Warsaw}
\begin{document}
\begin{frame}
\title{Number System}
\author{Shiv S. Dayal}
\titlepage
\end{frame}

\begin{frame}
\frametitle{Definitions}
Following is from \url{http://en.wikipedia.org/wiki/Numeral\_system}

A numeral system (or system of numeration) is a writing system for expressing 
numbers, that is, a mathematical notation for representing numbers of a given 
set, using digits or other symbols in a consistent manner. It can be seen as the 
context that allows the symbols ``11'' to be interpreted as the binary symbol 
for three, the decimal symbol for eleven, or a symbol for other numbers in 
different bases.

Ideally, a numeral system will:
\begin{enumerate}
\item Represent a useful set of numbers (e.g. all integers, or rational numbers)
\item Give every number represented a unique representation (or at least a 
standard representation)
\item Reflect the algebraic and arithmetic structure of the numbers.
\end{enumerate}

In our day-to-day use we use decimal number system while computers use binary. 
Then we can have complex numbers which we will deal with later and then we can 
have p-adic numbers which is out of scope for this book.
\end{frame}


\begin{frame}
\frametitle{Definitions}
\begin{enumerate}
 \item We use decimal number system which is a positional number system with 
base 10.
 \item It is made up of real numbers set which is denoted by $\mathbf{R}$.
 \item Real numbers consist of rational numbers and irrational numbers.
 \item Rational numbers consist of fractions, integers and zero denoted by 
$\mathbf{Q}$.
 \item Rational numbers are like $\frac{a}{b}$ where $a$ and $b$ are both 
integers and also $b\ne 0$.
 \item Integers are numbers like -3, -2, -1, 0, 1, 2, 3 and is denoted by set 
$\mathbf{I} \text{ or } \mathbf{Z}$.
\item Integers can be then further subdivided into positive and negative 
integers. 0 is neither positive not negative but is non-positive and 
non-negative and is considered an integer.
\item Positive integers are also called natural numbers and is denoted by set 
$\mathbf{N}$.
\end{enumerate}
\begin{figure}
\begin{center}
\caption{Number axis}
\begin{tikzpicture}
\draw (0,0) -- (7,0);
\draw (0,-0.2) -- (0,0.2);
\draw (7,-0.2) -- (7,0.2);
\draw (3.5,-0.2) -- (3.5,0.2);
\draw (0.4, 0.2) node {$-\infty$};
\draw (3.75, 0.2) node {$0$};
\draw (6.6, 0.2) node {$+\infty$};
\end{tikzpicture}
\end{center}
\end{figure}
\end{frame}

\begin{frame}
\frametitle{Definitions}
\begin{enumerate}
 \item Decimal form of rational number may not terminate. For example 
$\frac{5}{7}$ does not terminate but $\frac{1}{4}$ does.
 \item For standard form of rational number numerator and denominator have no 
factor common other than one and $b>0$. For example, $\frac{2}{4}$ is not in 
standard for but $\frac{1}{2}$ is in standard form.
\item While rational numbers can be written as a ratio an irrational number 
cannot be written as ratio for example $\pi \text{ or } e$.
\item Decimal form of rational number is neither terminating nor recurring i.e. 
we can say that it cannot be represented by a ratio. For example Ramanujan's 
famous formula for $\pi$ is
\begin{equation}
\frac{1}{\pi} = \frac{2\sqrt{2}}{9801} \sum^\infty_{k=0} 
\frac{(4k)!(1103+26390k)}{(k!)^4 396^{4k}} 
\end{equation}
As you see series is till $\infty$ so it never terminates.
\end{enumerate}
\end{frame}

\begin{frame}
 \frametitle{Representation as Sets}
 \begin{enumerate}
  \item Continuing we can say that N = \{1, 2, 3, 4, 5, ...\}
  \item I or Z = \{..., -3, -2, -1, 0, 1, 2, 3, ...\}
  \item $Q = \{\frac{a}{b}; a, b\in Z \text{ and } b \ne 0\}$
  \item $\in$ is the symbol used to denote that an element is \textit{in} some 
set.
  \item Clearly, $N\subset Z\subset Q\subset R$ where $\subset$ is the symbol 
for subset.
\end{enumerate}

\end{frame}

\end{document}

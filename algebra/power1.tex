\documentclass[aspectratio=1610]{beamer}
\usepackage{tikz, xcolor, amsmath}
\definecolor{webgreen}{rgb}{.3, .7, .3}
\color{webgreen}
%\setbeamercolor{structure}{fg=webgreen} 
%\color{webgreen}
\begin{document}
\begin{frame}
\title{Exponentiation by Squaring}
\subtitle{Basics for implementing math.h functions}
\author{Shiv S. Dayal}
\titlepage
\end{frame}

\begin{frame}
\frametitle{Objective}
\begin{itemize}
\item Our objective is to compute $x^y \text{ where } x, y \in \mathbf{I}$
where $\mathbf{I}$ is the set of integers.
\item The naive approach would be multiplying $x, y$ times.
\item To understand the approach it is important to understand that any integer
can be written as sum of powers of two. For example,
\item $1=2^0, 2=2^1, 3=1+2=2^0+2^1, 4=2^2, 5=1+4=2^0+2^2, 6=2+4=2^1+2^2$
\item $7=1+2+4=2^0+2^1+2^2, 8=2^3, 9=1+8=2^0+2^3 \text{ and so on}$
\item For example we have to compute $2^{25}$
\item $25=16+8+1=2^4+2^3+2^0$
\item So we square $2$ to get $4$ then we multiply that with $4$ to get $2^4$
and then and then again to get $2^8$ and then to get $2^{16}$. Now we multiply
$2^{16}, 2^8 \text { and } 2 $ to get $2^{25}$.
\end{itemize}
\end{frame}

\end{document}

\documentclass[aspectratio=1610]{beamer}
\usepackage{tikz, xcolor, amsmath}
\definecolor{webgreen}{rgb}{.3, .7, .3}
\color{webgreen}
%\setbeamercolor{structure}{fg=webgreen} 
%\color{webgreen}
\begin{document}
\begin{frame}
\title{Various Infinite Series}
\subtitle{Basics for math.h funcitons}
\author{Shiv S. Dayal}
\titlepage
\end{frame}

\begin{frame}
\frametitle{Various Infinite Series}
\begin{itemize}
\item We can represent trigonometric, lograthmic and exponential functions as
infinite expansions and then compute up to our desired precision.
\item The precision in C is fixed to float or double which may not be what we 
exactly want. So we will use GMP which is GNU's multiple precision arithmetic
library.
\item Read about GMP at \url{http://gmplib.org/}
\item First we will use only C then we will see examples using GMP.
\item To refresh the memory some expansions are given below:
\item Logrithm of $x$ with base $e$ or log natural of $|x|\le 1$ is given as
$ln(x)=log_e(x)=(x-1)-\frac{(x-1)^2}{2}+\frac{(x-1)^3}{3}-\frac{(x-1)^4}{4}+\frac{(x-1)^5}{5} ...$
\item $e^x$ is given by expansion as $e^x=1+x+\frac{x^2}{2!}+\frac{x^3}{3!}+
\frac{x^4}{4!}+ ....$
\item Sine and Cosine are given as $cos(x)=1-\frac{x^2}{2}+\frac{x^4}{4!}
-\frac{x^6}{6!} ...$
$sin(x)=x-\frac{x^3}{3!}+\frac{x^5}{5!}-\frac{x^7}{7!} ...$
\item We will start with \textit{lnx} and proceed so that we can compute
fractional powers as you will see.
\end{itemize}
\end{frame}

\begin{frame}
\frametitle{More on log series}
The series which we will use is for $|x|\ge 1$:

$ln\frac{x}{x-1} = \frac{1}{x} + \frac{1}{2x^2} + \frac{1}{3x^3} + ...$
so to compute for any natural number $y$ we have to substitute $x=\frac{y}{y-1}$

Let $lnx=y$ then $x=e^y.$ So, $x$ cannot be a real no. which is negative even if
$y$ is negative. Therefore, our input will always be positive and it is a
math domain error if it is negative. 
\end{frame}

\end{document}

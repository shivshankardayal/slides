\documentclass[aspectratio=1610]{beamer}
\usepackage{tikz, xcolor, amsmath, minted}
\definecolor{webgreen}{rgb}{.4,.7,.4}
\color{webgreen}
\setbeamercolor{structure}{fg=webgreen} 
\usetheme{Warsaw}
\begin{document}
\begin{frame}
\title{HTML Introduction}
\author{Shiv S. Dayal}
\titlepage
\end{frame}

\begin{frame}
\frametitle{Introduction}
HTML stands for \textbf{HyperText Markup Language}. HTML was initially
conceived and proposed by \textit{Tim Berners Lee}. At present world wide web
consortium is reposnsible body for HTML. Current specification work is going on
for HTML5. An HTML document describes content and how it will be presented.
It contsists of tags which are enclosed within angular brackets($<>$). Within the
tags content for the document is present.

\begin{enumerate}
\item Most of the tags come in pairs i.e. \texttt{<tag>content</tag>}.

\item HTML is typically used in conjunction with CSS and Javascript for better
looking user interfaces which provide more interactive and pleasant
interfaces. 

\item HTML pages are also called web pages and typically rendered by browsers.

\item Since HTML5 is going to be future we will discuss only HTML5 and not older
versions.
\end{enumerate}
\end{frame}

\begin{frame}[fragile]
\frametitle{A Sample HTML Document}
\begin{minted}{html}
<!DOCTYPE html>
<html>
<body>
<h1>A Heading</h1>
<p>A paragraph.</p>
</body>
</html>
\end{minted}
\begin{enumerate}
\item \texttt{<!DOCTYPE html>} is called the doctype declaration which helps the
browser identify the type of HTML document it is dealing with.

\item \texttt{<html></html>} is the root node of document.

\item \texttt{<body></body>} is the tag which contains visible content of the
  page.
\item \texttt{<h1></h1>} is one of the six headings page.
\item \texttt{<p></p>} is describes a paragraph.
\item The type of tag dectates how it will be rendered even when there is no
  CSS. Browser supplies a default stylesheet.
\end{enumerate}
\end{frame}

\end{document}
